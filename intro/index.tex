% Roman page numbering for preliminaries
\renewcommand{\thepage}{\arabic{page}}


\chapter{Introduction} % (fold)
\label{chap:intro}
\textit{Nota: 	Este guia de elaboração de relatório é apenas um guia não substituindo o diálogo necessário entre estudante e orientador para a melhor definição da estrutura e conteúdo do relatório em cada caso concreto.}

\vspace{2cm}

A introdução deve dar ao leitor a informação básica necessária por forma a facilitar o enquadramento dos objetivos, da abordagem seguida e dos resultados. A introdução começa com uma perspetiva geral do problema em estudo (na secção de enquadramento) em que, à medida que vai progredindo, deve ir fornecendo informação mais específica até se abordar a área em concreto tratada no relatório. Deve descrever, de forma sucinta, o problema em estudo e os objetivos. Deve também enunciar os principais métodos utilizados no trabalho, bem como a identificação clara dos contributos e aspetos inovadores da solução. A introdução deve terminar com a apresentação sucinta das secções que fazem parte do resto do relatório.


Nota: deve usar frases curtas; adotar o impessoal em vez do pessoal (e.g.\ adotou-se vs.\ adotei vs.\ adotamos); usar verbos conjugados no presente; evitar encadear verbos seguidos (e.g. “esta secção vai ser descrito” vs. “esta secção descreve”); usar voz passiva vs.\ ativa (ver Anexos).

\section{Enquadramento} % (fold)
\label{sec:enquadramento}
Enquadrar o problema no âmbito do projeto da organização proponente e descrever a motivação do estudante em aceitar este desafio. No caso de se tratar de um projeto proposto pelo estudante (sem estágio), descrever a motivação pessoal.

% section Enquadramento (end)

\section{Problem} % (fold)
\label{sec:problem}

O problema deve ser descrito de forma clara, de forma a que o leitor compreenda facilmente.

\subsection{Objectives} % (fold)
\label{sec:objectives}

Os objetivos devem ser descritos de forma detalhada devendo refletir a compreensão do estudante sobre o trabalho que foi desenvolvido.

% subsection Objetivos (end)

\subsection{Approach} % (fold)
\label{sec:approach}

A aproximação preconizada para a solução do problema ou do tratamento do tema focado, onde estejam claras as contribuições previstas. O objetivo é identificar a abordagem adotada (usando referência bibliográfica) e como foi aplicada ao projeto.

% subsection Abordagem (end)

\subsection{Contributions} % (fold)
\label{sec:contributions}

Devem ser apresentados os aspetos inovadores e de realce do trabalho, bem como a identificação dos benefícios trazidos para a organização (para a sociedade, para o ambiente,…). Devem ser identificadas as contribuições previstas (podendo usar estilo de apresentação por itens).

% subsection Contrubutos (end)

\subsection{Planning} % (fold)
\label{sec:planning}

% PODERÁ IR PARA ANEXO
Nesta subsecção deve ser apresentado o planeamento (cronograma, etc.) definido para a execução do projeto com indicação das tarefas, datas e principais fases (milestones).

% subsection Planeamento (end)

% section Descri\c{c}\~ao do Problema (end)

\section{Estrutura do Relat\'orio} % (fold)
\label{sec:estrutura}

Apresentação sucinta dos capítulos que fazem parte do relatório, descrevendo em poucos parágrafos o que cada um deles trata.

Para além da introdução, esta dissertação contém mais x capítulos. No capítulo 2, é descrito o estado da arte e são apresentados trabalhos relacionados. No capítulo 3, \ldots

% section Estrutura do Relatório (end)


% MAX 5 pags

% chapter Introdução (end)
