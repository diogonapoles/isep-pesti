\section{Problem} % (fold)
\label{sec:problem}

O problema deve ser descrito de forma clara, de forma a que o leitor compreenda facilmente.

\subsection{Objectives} % (fold)
\label{sec:objectives}

Os objetivos devem ser descritos de forma detalhada devendo refletir a compreensão do estudante sobre o trabalho que foi desenvolvido.

% subsection Objetivos (end)

\subsection{Approach} % (fold)
\label{sec:approach}

A aproximação preconizada para a solução do problema ou do tratamento do tema focado, onde estejam claras as contribuições previstas. O objetivo é identificar a abordagem adotada (usando referência bibliográfica) e como foi aplicada ao projeto.

% subsection Abordagem (end)

\subsection{Contributions} % (fold)
\label{sec:contributions}

Devem ser apresentados os aspetos inovadores e de realce do trabalho, bem como a identificação dos benefícios trazidos para a organização (para a sociedade, para o ambiente,…). Devem ser identificadas as contribuições previstas (podendo usar estilo de apresentação por itens).

% subsection Contrubutos (end)

\subsection{Planning} % (fold)
\label{sec:planning}

% PODERÁ IR PARA ANEXO
Nesta subsecção deve ser apresentado o planeamento (cronograma, etc.) definido para a execução do projeto com indicação das tarefas, datas e principais fases (milestones).

% subsection Planeamento (end)

% section Descri\c{c}\~ao do Problema (end)
