\begin{abstract}
O resumo do relatório (que só deve ser escrito após o texto principal do relatório estar completo) é uma apresentação abreviada e precisa do trabalho, sem acrescento de interpretação ou crítica, escrita de forma impessoal, podendo ter, por exemplo, as seguintes três partes:

\begin{enumerate}
    \item{} Um parágrafo inicial de introdução do contexto e do problema/objetivo do trabalho.
    \item{} Resumo dos aspetos mais importantes do trabalho descrito no presente relatório, que por sua vez documenta abordagem adotada e sistematiza os aspetos relevantes do trabalho realizado durante o estágio. Deve mencionar tudo o que foi feito, por isso deve concentrar-se no que é realmente importante e ajudar o leitor a decidir se quer ou não consultar o restante do relatório.
    \item{} Um parágrafo final com as conclusões do trabalho realizado.
\end{enumerate}

%----------------------------------------------------------------------------------------

\vspace*{10mm}
\noindent{}

\textbf{Palavras-chave (Tema)}: Incluir 3 a 6 palavras/expressões chave que caraterizem o
projeto do ponto de vista de tema/área de intervenção.

\vspace*{10mm}
\noindent{}

\textbf{Palavras-chave (Tecnologias)}: Incluir 3 a 6 palavras/expressões chave que
caraterizem o projeto do ponto de vista de tecnologias utilizadas.

\end{abstract}

